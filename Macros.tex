% ===========================
% Macro para Nombre Completo
% ===========================
% Este macro coloca el nombre completo en la parte superior del CV
% El primer argumento es para el nombre y el segundo para el apellido
\newcommand{\NombreCompleto}[2]{
	\node[above right] at ($(NW)+(\lm,-1.5)$) {
		\Large\bfseries 							% Estilo de fuente grande y en negrita
		{\color{ColorSecundario} #1} 				% Primer nombre en ColorSecundario
		{\color{white} #2} 							% Apellido en blanco
	};
}

% ===============================================================================================================

% ===========================
% Macro para Perfil
% ===========================
% Esta macro coloca el perfil en la parte superior del CV.
% El argumento #1 es el texto del perfil.
\newcommand{\Perfil}[1]{
	\node[white, below right] at ($(NW)+(\lm,-1.75)$) {
		\begin{minipage}{0.62\paperwidth} 					% Define un ancho para el texto
			{\large\bfseries PERFIL}\\ 						% Título "PERFIL" en grande y negrita
			#1 												% Texto del perfil proporcionado como argumento
		\end{minipage}
	};
}

% ===============================================================================================================

% ===========================
% Macro para Imagen de Perfil
% ===========================
% Esta macro coloca una imagen de perfil recortada en un círculo con un marco alrededor.
% El argumento #1 es el nombre del archivo de la imagen.
% El argumento #2 es la escala de la imagen (por ejemplo, 4cm).
\newcommand{\PerfilImagen}[2]{
	% Definir el radio del círculo basado en la escala
	\def\perfilradius{2.25cm}
	
	\begin{scope}
		% Recortar la imagen en un círculo
		\clip ($(NW)+(0.5*\lm,-0.5*\tm)$) circle [radius=\perfilradius];
		
		% Insertar la imagen centrada
		\node[inner sep=0pt] at ($(NW)+(0.5*\lm,-0.5*\tm)$) {\includegraphics[width=#2]{#1}};
	\end{scope}
	
	% Dibujar el marco alrededor de la imagen
	\draw[ColorSecundario, line width=1mm] ($(NW)+(0.5*\lm,-0.5*\tm)$) circle [radius=\perfilradius];
}

% ===============================================================================================================

% ===========================
% Macro para Profesión
% ===========================
% Esta macro coloca una franja dorada en la parte superior del CV con el texto centrado.
% El argumento #1 es el texto de la profesión o dedicación.
\newcommand{\Profesion}[1]{
	\fill[ColorSecundario] 
	($(NO)+(-4.5,-\tm)$) -- ++(0,0.5) 									% Línea vertical hacia abajo
	arc[start angle=180, end angle=90, radius=0.5] 						% Curva en la esquina superior derecha
	-- ++(9,0) 															% Línea horizontal derecha
	-- ++(0,0) 															% Extensión horizontal derecha (se puede omitir si es innecesario)
	arc[start angle=90, end angle=0, radius=0.5] 						% Curva en la esquina superior izquierda
	-- ++(0,-0.5) 														% Línea vertical hacia abajo
	-- ++(-9,0) 														% Línea horizontal inferior hacia la izquierda
	-- cycle; 															% Cierra la figura
	
	%% Texto centrado en la franja
	\node[white] at ($(NO)+({0.5},{0.5-\tm})$) {\bfseries\large #1};
}

% ===============================================================================================================
% ===============================================================================================================

% ===========================
% Macro para Contacto
% ===========================
\newcommand{\Contacto}[5]{
	% Dibuja la caja redondeada para el título de "CONTACTO"
	\draw[ColorSecundario, line width=0.5mm] 
	($(#1)+(-0.1,-#2)$) -- ++(6,0) 
	arc[start angle=90, end angle=-90, radius=0.5] 
	-- ++(-6,0) -- cycle;
	
	% Icono de información dentro de la caja de "CONTACTO"
	\node[ColorSecundario] at ($(#1)+(0.75,{-#2-0.5})$) {\Large\faInfoCircle};
	
	% Texto "CONTACTO" al lado del icono de información
	\node[white, right] at ($(#1)+(1.25,{-#2-0.5})$) {\Large\bfseries CONTACTO};
	
	% Íconos de contacto con bordes circulares y su información
	\node[white, draw=ColorSecundario, circle, line width=0.5mm, minimum size=1cm, inner sep=1pt] 
	at ($(#1)+(0.75,-2-#2)$) {\Large\faPhone}; 															% Teléfono
	\node[white, right] at ($(#1)+(1.5,-2-#2)$) {
		\begin{minipage}{5cm}
			{\bfseries TELÉFONO}\\
			#3
		\end{minipage}
	};
	\node[white, draw=ColorSecundario, circle, line width=0.5mm, minimum size=1cm, inner sep=1pt] 
	at ($(#1)+(0.75,-3.25-#2)$) {\Large\faEnvelopeO}; 													% Correo
	\node[white, right] at ($(#1)+(1.5,-3.5-#2)$) {
		\begin{minipage}{5cm}
			{\bfseries E-mail:}\\
			\href{E-mail:#4}{\textcolor{white}{#4}}  												% Hacer el correo clicable y en color blanco
		\end{minipage}
	};
	\node[white, draw=ColorSecundario, circle, line width=0.5mm, minimum size=1cm, inner sep=1pt] 
	at ($(#1)+(0.75,-4.5-#2)$) {\Large\faMapMarker}; 													% Dirección
	\node[white, right] at ($(#1)+(1.5,-4.75-#2)$) {
		\begin{minipage}{5cm}
			{\bfseries DIRECCIÓN:}\\
			#5
		\end{minipage}
	};
}

% ===============================================================================================================

% ===========================
% Macro para Competencias
% ===========================
% Esta macro crea la sección de competencias en la parte izquierda del CV.
\newcommand{\Competencias}[3]{
	% Dibuja la caja redondeada para el título de "COMPETENCIAS"
	\draw[ColorSecundario, line width=0.5mm] 
	($(#1)+(-0.1,-#2)$) -- ++(6,0) 
	arc[start angle=90, end angle=-90, radius=0.5] 
	-- ++(-6,0) -- cycle;
	
	% Ícono de engranajes para "COMPETENCIAS"
	\node[ColorSecundario] at ($(#1)+(0.75,{-#2-0.5})$) {\Large\faStar };
	
	% Texto "COMPETENCIAS" al lado del ícono de engranajes
	\node[white, right] at ($(#1)+(1.25,{-#2-0.5})$) {\large\bfseries COMPETENCIAS};
	
	% Lista de competencias
	\node[white, below right] at ($(#1)+(0,{-#2-0.5-0.75})$) {
		\begin{minipage}{6cm}
			\begin{itemize}[topsep=0pt,itemsep=3pt,partopsep=0pt,parsep=0pt]
				#3
			\end{itemize}
		\end{minipage}
	};
}

% ===============================================================================================================

% ===========================
% Macro para Idiomas
% ===========================
% Esta macro crea la sección de idiomas en la parte izquierda del CV.
\newcommand{\Idiomas}[3]{
	% Dibuja la caja redondeada para el título de "IDIOMAS"
	\draw[ColorSecundario, line width=0.5mm] 
	($(#1)+(-0.1,-#2)$) -- ++(6,0) 
	arc[start angle=90, end angle=-90, radius=0.5] 
	-- ++(-6,0) -- cycle;
	
	% Ícono de idioma para "IDIOMAS"
	\node[ColorSecundario] at ($(#1)+(0.75,{-#2-0.5})$) {\Large\faLanguage};
	
	% Texto "IDIOMAS" al lado del ícono de idioma
	\node[white, right] at ($(#1)+(1.25,{-#2-0.5})$) {\Large\bfseries IDIOMAS};
	
	% Lista de idiomas
	\node[white, below right] at ($(#1)+(0,{-#2-0.5-0.75})$) {
		\begin{minipage}{6cm}
			\begin{itemize}[topsep=0pt,itemsep=3pt,partopsep=0pt,parsep=0pt]
				#3
			\end{itemize}
		\end{minipage}
	};
}

% ===============================================================================================================

% ===========================
% Macro para Herramientas
% ===========================
% Esta macro crea la sección de herramientas en la parte izquierda del CV.
\newcommand{\Herramientas}[3]{
	% Dibuja la caja redondeada para el título de "HERRAMIENTAS"
	\draw[ColorSecundario, line width=0.5mm] 
	($(#1)+(-0.1,-#2)$) -- ++(6,0) 
	arc[start angle=90, end angle=-90, radius=0.5] 
	-- ++(-6,0) -- cycle;
	
	% Ícono para "HERRAMIENTAS"
	\node[ColorSecundario] at ($(#1)+(0.75,{-#2-0.5})$) {\Large\faWrench};
	
	% Texto "HERRAMIENTAS" al lado del ícono
	\node[white, right] at ($(#1)+(1.25,{-#2-0.5})$) {\large\bfseries HERRAMIENTAS};
	
	% Lista de herramientas
	\node[white, below right] at ($(#1)+(0,{-#2-0.5-0.75})$) {
		\begin{minipage}{6cm}
			\begin{itemize}[topsep=0pt,itemsep=3pt,partopsep=0pt,parsep=0pt]
				#3
			\end{itemize}
		\end{minipage}
	};
}


% ===============================================================================================================

% ===========================
% Macro para Intereses Personales
% ===========================
% Esta macro crea la sección de intereses en la parte izquierda del CV.
\newcommand{\Intereses}[3]{
	% Dibuja la caja redondeada para el título de "INTERESES"
	\draw[ColorSecundario, line width=0.5mm] 
	($(#1)+(-0.1,-#2)$) -- ++(6,0) 
	arc[start angle=90, end angle=-90, radius=0.5] 
	-- ++(-6,0) -- cycle;
	
	% Ícono de corazón con latido para "INTERESES"
	\node[ColorSecundario] at ($(#1)+(0.75,{-#2-0.5})$) {\Large\faSmileO};
	
	% Texto "INTERESES" al lado del ícono de corazón
	\node[white, right] at ($(#1)+(1.25,{-#2-0.5})$) {\Large\bfseries INTERESES};
	
	% Lista de intereses
	\node[white, below right] at ($(#1)+(0,{-#2-0.5-0.75})$) {
		\begin{minipage}{6cm}
			\begin{itemize}[topsep=0pt,itemsep=3pt,partopsep=0pt,parsep=0pt]
				#3
			\end{itemize}
		\end{minipage}
	};
}

% ===============================================================================================================
% ===============================================================================================================

% ===========================
% Macro para Formacion
% ===========================
% Esta macro crea la sección de educación en la parte derecha del CV con múltiples ítems.
\newcommand{\Experiencia}[2]{
	% Dibuja la caja redondeada para el título de "EXPERIENCIA"
	\def\abajo{#1}
	\draw[ColorPrimario, line width=0.5mm] ($(A)+(0,-\abajo)$) -- ++(8.5,0) arc[start angle=90, end angle=-90, radius=0.5] -- ++(-9,0) -- cycle;
	
	% Ícono para "EXPERIENCIA"
	\node[ColorPrimario] at ($(A)+(0.75,{-\abajo-0.5})$) {\Large\faSuitcase};
	
	% Texto "EXPERIENCIA" al lado del ícono
	\node[ColorPrimario, right] at ($(A)+(1.25,{-\abajo-0.5})$) {\Large\bfseries EXPERIENCIA};
	
	% Detalle de la experiencia
	\node[ColorPrimario, below right] at ($(A)+(0.5,{-\abajo-1.25})$) {
		\begin{minipage}{0.6\paperwidth}
			#2
		\end{minipage}
	};
}


% ===========================
% Macro para Formacion
% ===========================
% Esta macro crea la sección de educación en la parte derecha del CV con múltiples ítems.
\newcommand{\Formacion}[2]{
	% Dibuja la caja redondeada para el título de "FORMACION"
	\def\abajo{#1}
	\draw[ColorPrimario, line width=0.5mm] ($(A)+(0,-\abajo)$) -- ++(8.5,0) arc[start angle=90, end angle=-90, radius=0.5] -- ++(-9,0) -- cycle;
	
	% Ícono para "FORMACION"
	\node[ColorPrimario] at ($(A)+(0.75,{-\abajo-0.5})$) {\Large\faGraduationCap};
	
	% Texto "EDUCACION" al lado del ícono
	\node[ColorPrimario, right] at ($(A)+(1.25,{-\abajo-0.5})$) {\Large\bfseries FORMACION};
	
	% Detalle de la formacion
	\node[ColorPrimario, below right] at ($(A)+(0.5,{-\abajo-1.25})$) {
		\begin{minipage}{0.6\paperwidth}
			#2
		\end{minipage}
	};
}


% ===========================
% Macro para Certificaciones
% ===========================
% Esta macro crea la sección de certificaciones en la parte derecha del CV con múltiples ítems.
\newcommand{\Certificaciones}[3]{
	% Dibuja la caja redondeada para el título de "CERTIFICADOS"
	\def\abajo{#2}
	\draw[ColorPrimario, line width=0.5mm] 
	($(#1)+(0,-\abajo)$) -- ++(8.5,0) 
	arc[start angle=90, end angle=-90, radius=0.5] 
	-- ++(-9,0) -- cycle;
	
	% Ícono para "CERTIFICADOS"
	\node[ColorPrimario] at ($(#1)+(0.75,{-\abajo-0.5})$) {\Large\faCertificate};
	
	% Texto "CERTIFICADOS" al lado del ícono
	\node[ColorPrimario, right] at ($(#1)+(1.25,{-\abajo-0.5})$) {\Large\bfseries CERTIFICADOS};
	
	% Lista de certificaciones
	\node[ColorPrimario, below right] at ($(#1)+(0.5,{-\abajo-1.25})$) {
		\begin{minipage}{0.6\paperwidth}
			\begin{itemize}[topsep=0pt,itemsep=3pt,partopsep=0pt,parsep=0pt]
				#3
			\end{itemize}
		\end{minipage}
	};
}


% ===========================
% Macro para Proyectos Destacados
% ===========================
% Esta macro crea la sección de proyectos destacados en la parte derecha del CV con múltiples ítems.
\newcommand{\Proyectos}[3]{
	% Dibuja la caja redondeada para el título de "PROYECTOS"
	\def\abajo{#2}
	\draw[ColorPrimario, line width=0.5mm] 
	($(#1)+(0,-\abajo)$) -- ++(8.5,0) 
	arc[start angle=90, end angle=-90, radius=0.5] 
	-- ++(-9,0) -- cycle;
	
	% Ícono para "PROYECTOS"
	\node[ColorPrimario] at ($(#1)+(0.75,{-\abajo-0.5})$) {\Large\faListAlt};
	
	% Texto "PROYECTOS" al lado del ícono
	\node[ColorPrimario, right] at ($(#1)+(1.25,{-\abajo-0.5})$) {\Large\bfseries PROYECTOS};
	
	% Lista de proyectos destacados
	\node[ColorPrimario, below right] at ($(#1)+(0.5,{-\abajo-1.25})$) {
		\begin{minipage}{0.6\paperwidth}
			\begin{itemize}[topsep=0pt,itemsep=3pt,partopsep=0pt,parsep=0pt]
				#3
			\end{itemize}
		\end{minipage}
	};
}

% ===========================
% Macro para Voluntariado
% ===========================
% Esta macro crea la sección de voluntariado en la parte derecha del CV con múltiples ítems.
\newcommand{\Voluntariado}[3]{
	% Dibuja la caja redondeada para el título de "VOLUNTARIADOS"
	\def\abajo{#2}
	\draw[ColorPrimario, line width=0.5mm] 
	($(#1)+(0,-\abajo)$) -- ++(8.5,0) 
	arc[start angle=90, end angle=-90, radius=0.5] 
	-- ++(-9,0) -- cycle;
	
	% Ícono para "VOLUNTARIADOS"
	\node[ColorPrimario] at ($(#1)+(0.75,{-\abajo-0.5})$) {\Large\faGroup};
	
	% Texto "VOLUNTARIADOS" al lado del ícono
	\node[ColorPrimario, right] at ($(#1)+(1.25,{-\abajo-0.5})$) {\Large\bfseries VOLUNTARIADOS};
	
	% Lista de voluntariados
	\node[ColorPrimario, below right] at ($(#1)+(0.5,{-\abajo-1.25})$) {
		\begin{minipage}{0.6\paperwidth}
			\begin{itemize}[topsep=0pt,itemsep=3pt,partopsep=0pt,parsep=0pt]
				#3
			\end{itemize}
		\end{minipage}
	};
}


